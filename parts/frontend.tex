\section{Front end}
\label{sec:frontend}

Kao sto smo rekli, Front end je prva faza u kojoj se izvorni kod pretvara u Core jezik. Sastoji se od :
	 \begin{enumerate}
	 	\item Parsiranja
	 	\item Promena imena(eng. \emph{Rename}) 
	 	\item Provera tipa(eng. \emph{Typecheck})
	 	\item Prečišćavanja(eng. \emph{Desugaring})
	 \end{enumerate}


\subsection{Parsiranje}
\label{subsec:podnaslovParse}

Pravljenje preciznog parsera u konkretnom jeziku je jako teško. GHC-ov parser se služi sledećim principom : Često se parsira “previše velikodušno”,  a zatim odbacujemo loše slučajeve.
Paterni su parsirani kao izrazi i transformisani iz 
HsExpr.HsExp u HsPat.HsPat. Izraz kao sto je [ x | x <- xs] koji ne izgleda kao patern je odbijen.
Ponekad “previše velikodušno” parsiranje izvršava samo “renamer”. Na primer:
Infkiksni operatori  su parsirani kao da su svi levo asocijativni. “Renamer” koristi dekleracije ispravnosti za ponovno povezivanje sintaksnog stabla. Dobra karakteristika ovog pristupa je to da poruke o grešci kasnije  tokom kompilacije imaju tendenciju da pruže mnogo korisnije informacije. Greške generisane od strane samog parsera imaju tendenciju samo da kažu da se greška desila na odredjenoj liniji i ne pružaju nikakve dodatne informacije.

\subsection{Promena imena}
\label{subsec:podnaslovRename}

Osnovni zadatak Renamer-a je da zameni RdrNames sa Names. Na primer, imamo:

\begin{verbatim}
	module K where
	f x = True
	
	module N where
	import K
	
	module M where
	import N( f ) as Q
	f = (f, M.f, Q.f, \f -> f)
\end{verbatim}
U kome su sve promenljive tipa RdrName. Rezultat preimenovanja modula M je :
\begin{verbatim}
	M.f = (M.f, M.f, K.f, \f_22 -> f_22)
\end{verbatim} 
Gde su sada sve promenljive tipa Name.
\begin{itemize}
	\item Nekvantifikovani RdrName “f” na najvišem nivou postaje spoljašnji Name M.f.
	\item Pojavljivanja “f” i  “M.f” su zajedno vezane za ovo Name.
	\item  Kvantifikovani “Q.f” postaje Name “K.f” , zato što je funkcija definisana u modulu K.
	\item Lambda “f” postaje unutrašnji Name, ovde napisan f\_22.
\end{itemize}

Pored ovoga, renamer radi i sledeće stvari:
\begin{itemize}
	\item Vrši analizu zahteva za uzajmno rekurzivne grupe deklaracija. Ovo deli dekleracije u snažno povezane komponente.
	\item Izvršava veliki broj provera grešaka : promenljive van opsega, neiskorišćene biblioteke koje su uključene,..
	\item Renamer se nalazi izmedju parsera i typechecker-a, ipak, njegov rad je isprepletan sa typechecker-om.
\end{itemize}

\subsection{Provera tipa}
\label{subsec:podnaslovTypecheck}

Verovatno najvažnija faza u frontendu je kontrolor tipa (eng.\emph{type checker}), koji se nalazi u \underline{compiler/typecheck/}. GHC proverava  programe u njihovoj orginalnoj Haskel formi pre nego što ih desugar konvertuje u Core kod. Ovo umnogome komplikuje type checker ali poboljšava poruke o grešci.\\
GHC definise apstraktu sintaksu Haskel programa u \underline{compiler/hsSyn/hsSyn/HsSyn.hs} koristeći strukture koje apstraktuju konkretnu reprezentaciju graničnih pojavljivanja identifikatora i paterna. \\
Interfejs type checker-a ostatku kompajlera obezbeđuje \underline{compiler/typechecj/TcRnDriver.hs}. Svi moduli se izvršavaju zvanjem tcRnModule, i GHCI koristi tcRnStmt, tcRnStmt, tcRnExpr 
i tcRnType za proveru iskaza, izraza i tipova redom. \\
Funkcije tcRnModule i tcRnModuleTcRnM kontrolišu kompletnu statičku analizu Hasklel modula. Oni razrešavaju sve import iskaze, inicira stvarni postupak preimenovanja i provere tipa i završava sa obradom izvoza(eng. \emph{Export list}).\\
Reprezentacija tipova je fiksirana u modulu TypeRep i eksportovana kao podatak tipe Type.

\subsection{Prečišćavanje}
\label{subsec:podnaslovDesugar}

Prečišćavanje prevodi iz masivnog HsSyn tipa u GHC-ov medjujezik CoreSyn. Obicno se prečišćavanje programa izvršava pre faza povere tipa, ili preimenovanja, jer to onda olakšava posao renamer-u i typechecker-u jer imaju mnogo manji jezik za obradu.

\subsection{Core jezik}
\label{subsec:podnaslovCore}

Lepo bi bilo da pre nego što krenemo o optimizaijama napišemo nešto ukratko o Cor-u
