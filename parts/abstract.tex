\abstract
	Funkcionalni jezici ranije nisu bili popularni, pre svega zbog njihove manje upotrebe u industriji, a i zbog razvijenog mišljenja
	da su teški za učenje. Danas se oni sve više koriste i u neprekidnom su usponu, najviše u kritičnim delovima koda, 
	jer je mnogo lakše dokazati tačnost koda napisanog u funkcionalnom jeziku. U ovom radu, predstavićemo osnovne teorijske koncepte
	na kojima leže implementacije kompajlera funkcionalnih jezika, sa akcentom na Haskelu. Najviše se koncentrišemo na 
	proces prevođenja iz funkcionalnog koda u međujezik, i detaljno predstavljamo sve korake koji se nalaze u procesu prevođenja.