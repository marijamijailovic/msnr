\section{Back end}
\label{sec:backend}
\label{slike_i_tabele}


Kod prevodjenja jezika Core u neki imperativni medjujezik kao sto je C-- postoji vise etapa.
Prvo se CoreSyn (GHC’s intermediate language) prevodi u StgSyn (GHS’s intermediate language) I to u dve faze:


\paragraph{CoreToStg - }

Core-to-Core proces konvertuje program u ANF (A-normal form). A-normalnu formu su osmislili Sabry I Fellisen 1992. god.  U ANF formi svi argumenti moraju biti trivijalni. Odnosno  vrednost svih argumenata se mora izracunati odmah. ANF se bavi osnovnim definicijama zasnovanim na lamda(staviti znak) racunu sa slabom redukcijom I let izrazima uz ogranicenja:
- dozvoljene su samo konstante, lamda(staviti znak) termove I promenjive kao argumenti funkcije
- zahteva rezultat netrivijalnih izraza bude pripadati let-bound promenjivoj ili vracen iz funkcije \\ \\
Primer ANF: \\ f(g(x),h(y))\\

\begin{tabbing}
let v0 \= g(x) in \\
	\>let v1 \= h1(y) in \\
	\> \> f(v0, v1)
\end{tabbing}

\paragraph{CoreToStg - }

Rezultat prve faze u velikoj meri odgovara krajnjem StgSyn, zato u ovoj fazi nema preterano mnogo posla. Ova faza dekorise StgSyn sa mnogo pomocnih  promenjivih, let-no-escape indikatora.

STG program se pomocu  Code Generator-a pretvara u neki niži jezik kao što je C--.

\subsection{GHS faze}

Glasgov Haskel Compajler (\emph {Glasgow Haskell Compiler -GHC}) je u početku prevodio kod za STG-mašine (\emph{Spineless Tagless G-machine}) na C jezik. Ideja je bila da se ikoriste C kompajleri koji su portabilni i imaju određene dobre optimizacije. Međutim, pokazalo se da je C ima mnogo mana u kontekstu jezika srednjeg nivoa(\emph {intermediate language}), posebno za kompilatore lenjih funkcionalnih jezika sa nestandardnom kontrolom toka.Takođe C ne podržava repnu rekurziju, pristup steku radi čišćenja memorije(\emph{garbage collection}) i još mnogo drugih stvari. To nije iznenađujuće, jer C nije dizajniran za to. Pisci kompilatora višeg nivoa kao što je GHC su odlučili da ublaže prethodno navedene nedostatke. Problem su trebale da reše razne ekstenzije GNU C-a. Međutim i ova solucija ima svoje mane kao što su velike zavisnosti od GNU C verzije kompilatora, optimizacija za C često nema efekta na te jezike višeg nivoa - mnogo statičkih informacija bi moglo biti izgublljeno. Kao odgovor na to GHC je ubacio podršku za nejtiv kod generatore (\emph{native code generators}) koji direktno prevode u mašinski kod ali samo za određene sisteme, konkretno x86 , SPARC.



\begin{primer} Ovako se ubacuje slika. Obratiti pažnju da je dodato i 
\begin{verbatim}
\usepackage{graphicx}
\end{verbatim}

\begin{figure}[h!]
\begin{center}
%\includegraphics[scale=0.75]{panda.jpg}
\end{center}
\caption{Pande}
\label{fig:pande}
\end{figure}

Na svaku sliku neophodno je referisati se negde u tekstu. Na primer, na %slici \ref{fig:pande} prikazane su pande. 
\end{primer}

\begin{primer} I tabele treba da budu u svom okruženju, i na njih je neophodno referisati se u tekstu. Na primer, u tabeli% \ref{tab:tabela1} su prikazana različita poravnanja u tabelama.

\begin{table}[h!]
\begin{center}
\caption{Razlčita poravnanja u okviru iste tabele ne treba koristiti jer su nepregledna.}
\begin{tabular}{|c|l|r|} \hline
centralno poravnanje& levo poravnanje& desno poravnanje\\ \hline
a &b&c\\ \hline
d &e&f\\ \hline
\end{tabular}
\label{tab:tabela1}
\end{center}
\end{table}

\end{primer}